\documentclass[12pt]{article}

\usepackage{sty/sbc-template}

\usepackage{graphicx,url}

\usepackage[brazil]{babel}   
\usepackage[latin1]{inputenc}  

     
\sloppy

\title{Dados P�blicos \\ Enem 2013 }

\author{Ricardo Souza Rachaus\inst{1}, Juarez A. Sampaio Filho\inst{2}}


\address{
  Departamento de Ci�ncia da Computa��o\\
  Universidade de Bras�lia(UnB) -- Bras�lia, DF -- Brasil
  \email{juarez.asf@gmail.com, ricardorachaus@gmail.com }
}

\begin{document} 

\maketitle

\begin{resumo} 
  Dados p�blicos passaram a ser de grande interesse das pessoas com o advento da tecnologia, assim, muitos os buscam com diversos objetivos, desde estudos cient�ficos at� an�lises estat�sticas. Com o Enem n�o seria diferente, sendo o maior exame do pa�s, trabalhar sobre esses dados pode atrair o interesse de diversas �reas, e com seus dados p�blicos, torna-se uma �tima fonte de dados para pesquisas e trabalhos, como ser� visto neste texto.
\end{resumo}

\section{Introdu��o}
No in�cio, o Enem era apenas uma prova avaliativa do Ensino M�dio, por�m, hoje, � um dos principais instrumentos de sele��o para ingresso em institui��es de Ensino Superior. Assim, seus dados podem ter informa��es de grande utilidade para pesquisas.

Os microdados do Enem s�o o conjunto de dados obtidos ap�s sua aplica��o de um referido ano, contendo cidades, escolas, notas, dados de alunos e outros, sendo que nenhum destes dados implica na identifica��o direta de nenhum dos participantes.

Neste trabalho, ser� utilizado os dados do Enem referentes ao ano de 2013, armazendo estes em um Banco de Dados Relacional, mais especificadamente usando o SGBD PostgresSQL, onde as informa��es s�o armazenadas em tabelas e essas tabelas se relacionam. Com isso, busca-se obter informa��es interessantes sobre esses dados.

\section{Diagrama Entidade Relacionamento}
\section{Modelo Relacional}
\section{Formas Normais}
\section{Cria��o do Banco}
\section{ETL - Extract, Transform, Load}
\section{Visualiza��o dos Dados}
\section{View}
\section{Procedure}
\section{Trigger}
\section{Conclus�o}


\section{Bibliografia}

INSTITUTO NACIONAL DE ESTUDOS E PESQUISAS EDUCACIONAIS AN�SIO TEIXEIRA.
Microdados do Enem 2013. Bras�lia: Inep, 2015. Dispon�vel em: http://portal.inep.gov.br/basica-
levantamentos-acessar. Acesso em: 30 mai. 2015.

\bibliographystyle{sbc}
\bibliography{relatorio}

\end{document}
